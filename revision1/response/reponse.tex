% ======================================================= %
% Document: TEMPLATE FOR RESPONSES TO REVIEWERS
% Author: Andrea Ballatore
% Date: Jan 7, 2013
% Source: https://raw.githubusercontent.com/ucd-spatial/Datasets/master/tex_response_to_reviewers_template/responses_to_reviewers.tex
% Modified by Eduard Szöcs, 10.03.2015
% ======================================================= %
\documentclass[12pt]{article}

% packages
\usepackage{graphicx}
\usepackage{url}
\usepackage[usenames,dvipsnames]{xcolor}
\usepackage{color}
\definecolor{mygray}{gray}{0.6}
\usepackage[utf8]{inputenc}
\usepackage[onehalfspacing]{setspace}
\usepackage[
	round,	%(defaultage in the main file and \input ) for round parentheses;
	colon,	% (default) to separate multiple citations with colons;
	authoryear,% (default) for author-year citations;
	sort,		% orders multiple citations into the sequence in which they 
]{natbib}					
\usepackage[%disable
	]{todonotes}

\usepackage{anysize}
\marginsize{2.5cm}{2.5cm}{1.5cm}{2.5cm}

% macros
% add a counter
\newcounter{cN}
\setcounter{cN}{0}

\newcommand{\comment}[1]{
	\vspace{2em} 
	\refstepcounter{cN} % incrment counter
	\noindent \hangindent=0em \textbf{\textcolor{Maroon}{\uline{Comment \thecN}:~}}\emph{"#1"}
	}

\newcommand{\response}[1]{
	\\[0.25em] 
	\hangindent=2.3em \textbf{\textcolor{NavyBlue}{\uline{Response}:~}}#1 
	}

\usepackage[normalem]{ulem}
\definecolor{darkred}{rgb}{1,.6,.6}
\DeclareRobustCommand\problemline{\bgroup\markoverwith{\textcolor{darkred}{\rule[-0.9ex]{4pt}{3pt}}}\ULon}
\DeclareRobustCommand{\problem}[1]{\problemline{#1}} % soul
\setcounter{secnumdepth}{-1}

\begin{document}
% ======================================================= %
\title{Response to editorial team\\~\\JSS 2581: Szocs, Schafer \\ webchem: An R Package to Retrieve Chemical Information from the Web}

\author{Eduard Szöcs and Ralf B. Schäfer}

\maketitle
% ======================================================= %
\noindent Dear editorial team\\

We are thankful for pre-screening our manuscript to meet the requirements of JSS.
We made all required changes.  Moreover, we incorporated minor comments from a colleague.
Please find below detailed description of the changes made.

\vspace{2em}
\hfill Kind regards,

\hfill Eduard Szöcs and Ralf B. Schäfer
\newpage



% ======================================================= %

%done
\comment{Rather than using a goo.gl link, the official stable CRAN URL should be used: https://CRAN.R-project.org/package=webchem}
\response{We changed accordingly.}

%done
\comment{Title in title style (“retrieve” should be capitalized)}
\response{We changed accordingly.}

%done
\comment{section, subsection, etc. should be in sentence style (see http://www.jstatsoft.org/about/submissions), e.g., change “4. Use Cases” to “4. Use cases”}
\response{All sections etc. are not in sentence style.}

%done
\comment{The code presented in the manuscript should not contain comments within the verbatim code. Instead the comments should be made in the normal LaTeX text.}
\response{We removed all comments from the code blocks and clarified the text.}

%done
\comment{For the code layout in R publications, the input should use the text width (up to 76 or 77 characters) and be indented by two spaces, e.g.,}
\response{We fixed the width to 76 characters and removed excessive indentation.}

%done
\comment{For R-related manuscripts: The first argument of data() and library() should always be quoted, e.g., library("foo"). [' library(webchem)', ' data(jagst)', ' data(lc50)']}
\response{We changed accordingly.}

\comment{ If using "e.g." and "i.e." add a comma after the period to keep LaTeX from interpreting them as the end of a sentence, i.e.: "e.g., " and "i.e., ".}
\response{We added a backslash ("i.e.\textbackslash " or "e.g.\textbackslash"), which also prevents this behaviour.}

\comment{There should not be further footnote-style annotations in tables; these should all be placed in the caption.}
\response{We removed the footnote-style annotation from table 1.}

\comment{As a reminder, please make sure that: proglang, pkg and code have been used for highlighting throughout the paper (including titles and references), except where explicitly escaped.}
\response{We checked thoroughly the manuscript.}

\comment{Springer-Verlag (not: Springer)}
\response{We changed accordingly.}

\comment{Please make sure that all software packages are cited properly.}
\response{We checked thoroughly and updated the citations. }

\comment{Please make sure that the files needed to replicate all code/examples within the manuscript are included in a standalone replication script.}
\response{These are given in the accompanying file "article.R".}

%% --------------------------------
% \newpage
% \bibliography{refs}
% \bibliographystyle{spbasic}


% ======================================================= %
\end{document}
% ======================================================= % 
